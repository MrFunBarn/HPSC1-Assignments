\documentclass{IEEEtran}
\usepackage{graphicx}


\begin{document}

\title{Final Project Idea: Hunting Awesome Variable Stars in Sloan Digital Sky
Survey Data}
\author{Brandon Bell}
%\markboth{Brandon}{HW 2}
\maketitle
\begin{abstract}
I would like to hunt for a rare type of variable star called a Luminous Blue
Variable (LBV) in Sloan Digital Sky Survey (SDSS) data with an Artificial
Neural Network.
\end{abstract}

\section{Luminous Blue Variables}
LBVs are are very large, about 60 times the mass of the sun, in the late stages
of their journey towards exploding as a super novae. LBVs exhibit large
variability in brightness and color accompanied with drastic spectral changes;
this variability is thought to be connected with large stellar pulsations the
can result in the ejection of entire layers of the star into space. It is
through these eruptions that the stars loose much of their mass to the
surrounding space creating large infared Nebulae around them. The exact nature
and physics of these eruptions is not well understood at this time. One of the
principal causes of our current ignorance is the extreme rarity of these
objects; the LBV phase is a short lived phase of evolution of only the largest
stars which are themselves, very rare. There only about 16 confirmed LBVs known
with a slew of some 30 possible LBVs. Locating more LBVs in either our galaxy
or, in nearby galaxies, is an import issue in the study of  massive stellar
evolution and import to our understanding of the physics driving very large
stars in general. SDSS is a large imaging and specrographic survey of thousands
of stars in both the visible and infrared with data on hundreds of stellar
parameters of each star and I hope to train an artificial nueral network to
find features unique to LBVs and hopeful find new LBVs hidden in the mountain
of SDSS data.

\section{The Code }
For the code, I’ve found two C libraries implementing various forms of Neural
networks ( Darknet, FANN ) but, I would like to try and scale, or implement a
new, Scikit-Learn python routine for training a back-propagating neural network
in a parallel, perhaps with openMP or GPGPU support. Scikit-learn’s Multi-layer
neural network is explicitly “not intended for large scale applications … as it
offers no GPU support” (scikit-learn.org). I could implement large scale
parallel support for training with pythonMPI or, write c code to train on the
data and simply produce a weights file for use Scikit-Learn, or Write the c
code and a cython wrapper to integrate directly with the python code.

\end{document}
